\documentclass[../main.tex]{subfiles}
\begin{document}

Система типов $\lambda_{\wedge \eta}$ отличается от просто типизированного лямбда исчисления новым типовым конструктором: $\wedge$, соответствующим пересечению типов. Эту операцию можно понимать в теоретико-множественном смысле: типом $\sigma \wedge \tau$ типизируются такие и только такие термы, которые типизируются и $\sigma$, и $\tau$. Правила вывода, соответствующие этому поведению, обозначаются $(I\wedge)$ и $(E\wedge)$ (введение пересечения и элиминация пересечения соответственно). 

Кроме того, вводится ещё одно дополнительное правило $(\eta)$, позволяющее проводить эта-экспансию. 

Таким образом, система состоит из следующих правил вывода: $Ax$, $I\to$, $E\to$, $I\wedge$, $E\wedge$, $\eta$.
\begin{equation} \label{rules eta}
\begin{aligned}
\infer[(Ax)]{\Gamma, x \colon \tau \vdash x \colon \tau}{}\\[2pt]
\infer[(I\to)]{\Gamma \vdash (\lambda x . M) \colon \sigma \to \tau}{\Gamma, x \colon \sigma \vdash M \colon \tau}\\[2pt]
\infer[(E\to)]{\Gamma \vdash (MN) \colon \tau}{\Gamma \vdash M \colon \sigma \to \tau & \Gamma \vdash N \colon \sigma}\\[2pt]
\infer[(I\wedge)]{\Gamma \vdash M \colon \sigma \wedge \tau}{\Gamma \vdash M \colon \sigma & \Gamma \vdash M \colon \tau}\\[2pt]
\infer[(E\wedge)]{\Gamma \vdash M \colon \sigma}{\Gamma \vdash M \colon \sigma \wedge \tau}\\[2pt]
\infer[(\eta)]{\Gamma \vdash (\lambda x . M x) \colon \sigma}{\Gamma \vdash M \colon \sigma}

\end{aligned}
\end{equation}

\subsection{Подтипизация}


Определим на типах отношение подтипизации. С теоретико-множественной точки зрения подтпизация соответствует отношению <<быть подмножеством>>: если терм типизируется $\sigma$, то он типизируется всеми надтипами $\sigma$, то есть такими $\tau$, что $\sigma \leqslant \tau$. Отношение определим следующими аксиомами и правилами, аналогично определению из \cite{hindley_1982}.

\begin{equation} \label{rules leq}
\begin{aligned}
\infer[(A1)]{\sigma \leqslant \sigma}{} \\[2pt]
\infer[(A2)]{\sigma \leqslant \sigma \wedge \sigma}{} \\[2pt]
\infer[(A3)]{\sigma \wedge \tau \leqslant \sigma}{} \\[2pt]
\infer[(A4)]{\sigma \wedge \tau \leqslant \tau}{} \\[2pt]
\infer[(A5)]{(\sigma \to \tau_1) \wedge (\sigma \to \tau_2) \leqslant \sigma \to (\tau_1 \wedge \tau_2)}{} \\[2pt]
\infer[(R1)]{\sigma \wedge \tau \leqslant \sigma' \wedge \tau'}{\sigma \leqslant \sigma' & \tau \leqslant \tau'} \\[2pt]
\infer[(R2)]{\sigma' \to \tau \leqslant \sigma \to \tau'}{\sigma \leqslant \sigma' & \tau \leqslant \tau'} \\[2pt]
\infer[(R3)]{\tau_1 \leqslant \tau_3}{\tau_1 \leqslant \tau_2 & \tau_2 \leqslant \tau_3} 
\end{aligned}
\end{equation}


В \cite{hindley_1992} показано, что правило $(\eta)$ может быть заменено следующим правилом:

$$\infer[(\leqslant)]{\Gamma \vdash M : \tau}{\Gamma \vdash M : \sigma & \sigma \leqslant \tau}$$

На самом деле, поскольку $\sigma \wedge \tau \leqslant \sigma$ и $\sigma \wedge \tau \leqslant \tau$, правило $(E \wedge)$ также избыточно. Таким образом, правила в этой системе следующие: 

\begin{equation} \label{rules subtype}
\begin{aligned}
\infer[(Ax)]{\Gamma, x \colon \tau \vdash x \colon \tau}{}\\[2pt]
\infer[(I\to)]{\Gamma \vdash (\lambda x . M) \colon \sigma \to \tau}{\Gamma, x \colon \sigma \vdash M \colon \tau}\\[2pt]
\infer[(E\to)]{\Gamma \vdash (MN) \colon \tau}{\Gamma \vdash M \colon \sigma \to \tau & \Gamma \vdash N \colon \sigma}\\[2pt]
\infer[(I\wedge)]{\Gamma \vdash M \colon \sigma \wedge \tau}{\Gamma \vdash M \colon \sigma & \Gamma \vdash M \colon \tau}\\[2pt]
\infer[(\leqslant)]{\Gamma \vdash M : \tau}{\Gamma \vdash M : \sigma & \sigma \leqslant \tau}

\end{aligned}
\end{equation}

Полученную систему будем называть $\lambda_{\wedge \leqslant}$. Она эквивалентна $\lambda_\wedge_\eta$ в смысле типизации: утверждение типизации, верное в системе  $\lambda_{\wedge \leqslant}$, верно в $\lambda_\wedge_\eta$ и наоборот. Поэтому в контексте нашей задачи эти две системы полностью взаимозаменяемы.

Определим отношение эквивалентности на типах следующим образом. 

\begin{definition}
$\sigma \sim \tau \iff \sigma \leqslant \tau$ и $\tau \leqslant \sigma$
\end{definition}

Благодаря правилу $(\leqslant)$, для эквивалентных типов верно следующее утверждение:

\begin{lemma}
Множества термов, тпизируемых эквивалентными типами, в точности совпадают.
\end{lemma}

В частности, это означает, что переход к эквивалентному типу не влияет на его обитаемость.

Легко видеть, что отношение <<$\sim$>> является отношением конгруэнтности (в смысле \cite{barendregt_2013}) а именно, верна следующая Лемма:
\begin{lemma}
Пусть $\alpha, \beta, \gamma, \delta$ -- типы. 
Если $\alpha \sim \beta$ и $\gamma \sim \delta$, то $(\alpha \to \gamma) \sim (\beta \to \delta)$, а также $(\alpha \wedge \gamma) \sim (\beta \wedge \delta)$.
\end{lemma}



Для удобства введём следующее обозначение: 
\begin{definition}
\begin{align*}
\hat{\tau} = \{\tau_i \mid & \tau_1 \wedge \dots \wedge \tau_k = \tau \\
                               && \text{и $\tau_i$ не является пересечением ни для какого i} \}
\end{align*}
То есть $\hat{\tau}$ -- множество, полученное разделением всех пересечений на верхнем уровне $\tau$.
\end{definition}


В \cite{hindley_1982} показана следующая экивалентность: 
\begin{lemma}
$\alpha \to \beta \wedge \gamma \sim (\alpha \to \beta) \wedge (\alpha \to \gamma)$
\end{lemma}

Применяя её многократно, мы можем привести тип к нормальной форме. А именно, введём операцию <<*>> следующим образом: 

\begin{definition}
\begin{align*}
\alpha^* &= \alpha \\
    &\text{если $\alpha$ --- типовая переменная }\\
(\sigma \wedge \tau)^* &= \sigma^* \wedge \tau^*\\
(\sigma \to \tau)^* &= \bigwedge \limits_{\varphi \in \hat{\tau^*}} \sigma \to \varphi
\end{align*}
\end{definition}

Например, $(\alpha \to \gamma \wedge (\beta \to \gamma \wedge \delta))^* = (\alpha \to \beta \to \gamma) \wedge (\alpha \to \beta \to \delta) \wedge (\alpha \to \gamma)$.

Операция <<*>> переводит тип в эквивалентный.

\begin{lemma} \label{normal form}
Для любого типа $\tau$: $\tau \sim \tau^*$
\end{lemma}


\begin{definition}
Тип $\tau$ находится в {\it нормальной форме}, если $\tau^* = \tau$.
\end{definition}

Общий вид типа в нормальной форме после применения <<*>> описывается следующей леммой:

\begin{lemma}
Для любого типа $\tau$: $\tau^* = \bigwedge \limits_i (\varphi_i_1 \to \cdots \to \varphi_i_{k_i} \to \alpha_i$), где $\alpha_i$ - типовая переменная.
\end{lemma}

\begin{corollary} \label{normal form unit}
Типы в нормальной форме без пересечений на верхнем уровне --- в точности типы вида $\dots \to \alpha$, где $\alpha$ -- некоторая типовая переменная.
\end{corollary}

В \cite{hindley_1982} показано, что отношение $(\leqslant)$ рекурсивно и существует алгоритм, позволяющий определить для двух типов $\alpha$ и $\beta$ верно ли, что $\alpha \leqslant \beta$:


\begin{algorithm} \label{subt algo}
\begin{align*}
    \alpha &\leqslant \beta  = (\alpha \ttt{ == } \beta) \\ 
                            &\text{если $\alpha$ и $\beta$ --- типовые переменные} \\
    \sigma &\leqslant \tau = False \\ 
                          &\text{если один из типов --- переменная, а другой тип стрелочный}\\
    (\sigma_1 \to \tau_1) &\leqslant (\sigma_2 \to \tau_2) = (\tau_1 \leqslant \tau_2) \ttt{ AND } (\sigma_2  \leqslant \sigma_1) \\
    \sigma &\leqslant \tau = \forall \tau_i \in \hat{\tau^*} \colon \exists \sigma_j \in \hat{\sigma^*} \colon \sigma_j \leqslant \tau_i
\end{align*}
\end{algorithm}

Введём ещё несколько обозначений. 

\begin{definition}
Обозначим множество всех подтипов $\tau$ через $\ub{\tau}$, а множество всех его надтипов через $\ob{\tau}$. То есть $\ub{\tau} = \{ \sigma \mid \sigma \leqslant \tau \}$, $\ob{\tau} =  \{ \sigma \mid \tau \leqslant \sigma \}$. 
\end{definition}

Далее будем считать, что надтипы и подтипы могут использоваться везде, где могут использоваться типы. При этом операции над типами <<поднимаются>> на уровень множеств. Так, например, выражение $\alpha \to \ob{\beta} \to \ub{\gamma}$ следует понимать как множество $\{ \alpha \to \beta' \to \gamma' \mid \beta' \in \ob{\beta}, \gamma' \in \ub{\gamma} \}$.

Утверждение о типизации, в котором фигурируют надтипы или подтипы, будем считать выводимым, если оно выводимо {\it для некоторых} элементов соответствующих можеств надтипов и подтипов.

Запись о вхождении типизации в контекст $(x \colon T) \in \Gamma$ в случае, если в $T$ фигурируют надтипы или подтипы, означает, что $(x \colon \tau) \in \Gamma$ {\it для некоторого} $\tau \in T$.

Запись вида $T_1 \leqslant T_2$, где в $T_i$ могут фигурировать подтипы и надтипы, означает, что $\forall \tau_1 \in T_1, \tau_2 \in T_2 : \tau_1 \leqslant \tau_2$.

\begin{lemma} \label{подтип стрелки}
Пусть $\sigma \to \tau$ -- тип в нормальной форме. Тогда
$\ub{\sigma \to \tau} = \varphi \wedge (\ob{\sigma} \to \ub{\tau})$,
для некоторого (возможно, пустого) $\varphi$.

\end{lemma}

\begin{proof}
    Включение $\varphi \wedge (\ob{\sigma} \to \ub{\tau}) \subseteq \ub{\sigma \to \tau}$ почти очевидно: 
    $$\infer[(A4)]{\varphi \wedge (\ob{\sigma} \to \ub{\tau}) \leqslant \sigma \to \tau}
                  {\infer[(R2)]{\ob{\sigma} \to \ub{\tau}}
                               {\sigma \leqslant \ob{\sigma} & \ub{\tau} \leqslant \tau}}$$
                               
    Для включения $\ub{\sigma \to \tau} \subseteq \varphi \wedge (\ob{\sigma} \to \ub{\tau})$ достаточно посмотреть на Алгоритм~\ref{subt algo}. Подтип $\sigma \to \tau$ --- это либо тип из $(\ob{\sigma} \to \ub{\tau})$ (третий случай алгоритма), либо тип-пересечение, один из элементов которого --- подтип $\sigma \to \tau$; остальные элементы можно <<переместить>> в $\varphi$ перестановкой.
\end{proof}


\subsection{Существенность $\eta$ правила}

Насколько добавление $\eta$ - правило меняет систему? 
Система $\lambda_\wedge$ существенно слабее $\lambda_\wedge_\eta$, а именно, есть тип, необитаемый в системе без эта-правила и обитаемый в системе с ним. 

\begin{lemma} \label{существенность: типизация}
Утверждение о типизации $x : \alpha \to \beta \wedge \gamma \vdash x : \alpha \to \beta$ верно в $\lambda_{\wedge \eta}$ но неверно в $\lambda_{\wedge}$.
\end{lemma}

\begin{proof}
Докажем выводимость в системе $\lambda_{\wedge \leqslant}$:

$$\infer{x \colon \alpha \to \beta \wedge \gamma \vdash x \colon \alpha \to \beta}
        {\infer{x \colon \alpha \to \beta \wedge \gamma \vdash x \colon \alpha \to \beta \wedge \gamma} &
            \alpha \to \beta \wedge \gamma \leqslant \alpha \to \beta}
         $$ 
Чтобы доказать, что утверждение неверно в системе $\lambda_{\wedge}$, предположим обратное. Рассмотрим последнее правило вывода, которое могло быть применено, чтобы получить данное утверждение. $(I \to)$ и $(E \to)$ не могли быть применены, поскольку они типизируют абстракцию и аппликацию соответственно, а $x$ --- типовая переменная. Правило $(Ax)$ требует наличия соответствующей типизации в контексте, правило $(I \wedge)$ приписывает терму тип-пересечение, а не стрелочный тип.

Таким образом, единственным возможным правилом могло быть $(E \wedge)$:

$$\infer{x \colon \alpha \to \beta \wedge \gamma \vdash x \colon \alpha \to \beta}
        {\infer*{x \colon \alpha \to \beta \wedge \gamma \vdash x \colon (\alpha \to \beta) \wedge \sigma}}
$$ 
Какие правила вывода могли привести к данному утверждению о типизации? Только $(I\to)$ и $(E\to)$. Однако легко видеть, что их <<обратное>> применение порождает утверждение вида $x \colon \alpha \to \beta \wedge \gamma \vdash x \colon (\alpha \to \beta) \wedge \sigma$ для некоторого (возможно, пустого) $\sigma$, но утверждение такого вида уже встречалось ранее, и могло быть получено лишь с помощью $(I\to)$ и $(E\to)$. Значит, дерева вывода не существует.

\end{proof}

\begin{lemma}
Тип $\delta \wedge (\alpha \to \beta \wedge \gamma) \to \delta \wedge (\alpha \to \beta)$ пуст в $\lambda_{\wedge}$, но содержит терм \ttt{id} в $\lambda_{\wedge \eta}$. 
\end{lemma}

\begin{proof}

Анализом возможных деревьев вывода, аналогичным рассуждениям в предыдущей лемме, получаем, что для существования типизации $\vdash M \colon \delta \wedge (\alpha \to \beta \wedge \gamma) \to \delta \wedge (\alpha \to \beta)$ необходима и достаточна типизация $x \colon \alpha \to \beta \wedge \gamma \vdash x \colon \alpha \to \beta$. 

\end{proof}

\begin{notice}
    Тип  $(\alpha \to \beta \wedge \gamma) \to \alpha \to \beta$ обитаем в обеих системах. В $\lambda_{\wedge}$ он содержит $\lambda a b . a b$, но не $\lambda a . a$.
\end{notice}

\subsection{Выводы и результаты по главе}

В данной главе была рассмотрена система типов $\lambda_{\wedge \eta}$ и эквивалентная ей $\lambda_{\wedge \leqslant}$, для которых в следующих разделах будет приведён и доказан населяющий алгоритм. Также было введено отношение подтипизации <<$\leqslant$>>; отношение эквивалентности <<$\sim$>>; операции $\ob{\tau}$ и $\ub{\tau}$, ставящие в соответствие типу множество его надтипов и подтипов соответственно;  операция <<$*$>>, приводящая тип к нормальной форме. Алгоритм приведения к нормальной форме является важной составной частью населяющего алгоритма, описанию которого посвящена следующая глава.


\end{document}
