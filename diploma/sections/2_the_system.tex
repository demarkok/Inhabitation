\documentclass[../main.tex]{subfiles}
\begin{document}\label{sec:the_system}

Термы в системе $\lambda_{\wege \eta}$ устроены так же, как в просто типизированном лямбда исчислении в стиле Карри \cite{curry_1969}. Их грамматика выглядит следующим образом:
\begin{definition}[Термы в системе $\lambda_{\wedge \eta}$]
\begin{align*}
t ::= & \text{ } x              && \text{переменная, $x \in S$}\\
      & \text{ } \lambda x . t  && \text{абстракция, $x \in S$}\\
      & \text{ } t t            && \text{аппликация}
\end{align*}
Здесь $S$ -- некоторое заданное бесконечное множество имён переменных.
\end{definition}


Типы в системе $\lambda_{\wedge \eta}$ отличается от просто типизированного лямбда исчисления добавлением новго типового конструктора: $\wedge$, соответствующего пересечению типов.

\begin{definition}[Типы в системе $\lambda_{\wedge \eta}$]
\begin{align*}
T ::= & \text{ } \alpha          && \text{типовая переменная (атом), $\alpha \in \Sigma$}\\
      & \text{ } T \to T         && \text{тип функции}\\
      & \text{ } T \wedge T      && \text{тип-пересечение}\\
\end{align*}
Здесь $\Sigma$ -- некоторое заданное бесконечное множество имён типовых переменных.
\end{definition}

Подразумевается, что операция <<$\wedge$>> ассоциативна, коммутативна и идемпотентна, и её приоритет выше, чем у <<$\to$>>, то есть $\rho \to \sigma \wedge \tau$ -- это переобозначение для $\rho \to (\sigma \wedge \tau)$, а $\rho \wedge \sigma \to \tau$ -- переобозначение для $(\rho \wedge \sigma) \to \tau$

Далее греческими буквами из начала алфавита ($\alpha, \beta, \gamma, \dots$) будут обозначаться типовые переменные, а греческими буквами из конца ($\pi, \rho, \sigma, \tau, \dots$) -- произвольные типы. Маленькими латинскими буквами обозначаются термовые переменные, а большими латинскими буквами -- метапеременные, соответствующие произвольным термам системы.


Операцию <<$\wedge$>> можно понимать в теоретико-множественном смысле: типом $\sigma \wedge \tau$ типизируются такие и только такие термы, которые типизируются и $\sigma$, и $\tau$. Правила вывода, соответствующие этому поведению, обозначаются $(I\wedge)$ и $(E\wedge)$ (введение пересечения и удаление пересечения соответственно). 

Кроме того, вводится ещё одно дополнительное правило $(\eta)$, позволяющее проводить эта-редукцию. 

Таким образом, система состоит из следующих правил вывода: $Ax$, $I\to$, $E\to$, $I\wedge$, $E\wedge$, $\eta$.

\begin{definition}[Правила вывода системы $\lambda_{\wedge \eta}$]
\begin{mathpar}
\infer[(Ax)]{\Gamma, x \colon \tau \vdash x \colon \tau}{}\\
\infer[(I\to)]{\Gamma \vdash (\lambda x . M) \colon \sigma \to \tau}{\Gamma, x \colon \sigma \vdash M \colon \tau} \\
\infer[(E\to)]{\Gamma \vdash (MN) \colon \tau}{\Gamma \vdash M \colon \sigma \to \tau & \Gamma \vdash N \colon \sigma} \\
\infer[(I\wedge)]{\Gamma \vdash M \colon \sigma \wedge \tau}{\Gamma \vdash M \colon \sigma & \Gamma \vdash M \colon \tau} \\
\infer[(E\wedge)]{\Gamma \vdash M \colon \sigma}{\Gamma \vdash M \colon \sigma \wedge \tau}\\
\infer[(\eta)]{\Gamma \vdash M \colon \sigma}{\Gamma \vdash (\lambda x . M x) \colon \sigma}
\end{mathpar}
\end{definition}

Стоит заметить, что <<парное>> правило для $(E \wedge)$ не нужно в силу коммутативности <<$\wedge$>>.

\subsection{Подтипизация}

Рассматриваемая система типов основывается на просто типизированном лямбда исчислении в стиле Карри \cite{curry_1969}, поэтому один и тот же терм может типизироваться разными способами. Так, например, $\lambda x . x$ типизируется как типом $\alpha \to \alpha$, так и типом $\beta \to \beta$, а значит, и $(\alpha \to \alpha) \wedge (\beta \to \beta)$. Это мотивирует ввести на множестве типов отношение подтипизации. 

С теоретико-множественной точки зрения подтпизация соответствует отношению <<быть подмножеством>>: если терм типизируется $\sigma$, то он типизируется всеми надтипами $\sigma$, то есть такими $\tau$, что $\sigma \leqslant \tau$. Отношение определим следующими аксиомами и правилами, аналогично определению из \cite{hindley_1982}.
% \newpage

\begin{definition}[Правила вывода отношения подтипизации]
\begin{mathpar}
\infer[(A1)]{\sigma \leqslant \sigma}{} \\
\infer[(A2)]{\sigma \leqslant \sigma \wedge \sigma}{} \\
\infer[(A3)]{\sigma \wedge \tau \leqslant \sigma}{} \\
\infer[(A4)]{\sigma \wedge \tau \leqslant \tau}{} \\
\infer[(A5)]{(\sigma \to \tau_1) \wedge (\sigma \to \tau_2) \leqslant \sigma \to (\tau_1 \wedge \tau_2)}{} \\
\infer[(R1)]{\sigma \wedge \tau \leqslant \sigma' \wedge \tau'}{\sigma \leqslant \sigma' & \tau \leqslant \tau'} \\
\infer[(R2)]{\sigma' \to \tau \leqslant \sigma \to \tau'}{\sigma \leqslant \sigma' & \tau \leqslant \tau'} \\
\infer[(R3)]{\tau_1 \leqslant \tau_3}{\tau_1 \leqslant \tau_2 & \tau_2 \leqslant \tau_3} 
\end{mathpar}
\end{definition}


В \cite{hindley_1992} показано, что правило $(\eta)$ может быть заменено следующим правилом:

$$\infer[(\leqslant)]{\Gamma \vdash M : \tau}{\Gamma \vdash M : \sigma & \sigma \leqslant \tau}$$

На самом деле, поскольку $\sigma \wedge \tau \leqslant \sigma$ и $\sigma \wedge \tau \leqslant \tau$, правило $(E \wedge)$ также избыточно. Таким образом, правила в этой системе следующие: 

\begin{definition}[Правила вывода системы $\lambda_{\wedge \leqslant}$]
\begin{mathpar}
\infer[(Ax)]{\Gamma, x \colon \tau \vdash x \colon \tau}{}\\
\infer[(I\to)]{\Gamma \vdash (\lambda x . M) \colon \sigma \to \tau}{\Gamma, x \colon \sigma \vdash M \colon \tau}\\
\infer[(E\to)]{\Gamma \vdash (MN) \colon \tau}{\Gamma \vdash M \colon \sigma \to \tau & \Gamma \vdash N \colon \sigma}\\
\infer[(I\wedge)]{\Gamma \vdash M \colon \sigma \wedge \tau}{\Gamma \vdash M \colon \sigma & \Gamma \vdash M \colon \tau}\\
\infer[(\leqslant)]{\Gamma \vdash M : \tau}{\Gamma \vdash M : \sigma & \sigma \leqslant \tau}
\end{mathpar}
\end{definition}


Полученную систему будем называть $\lambda_{\wedge \leqslant}$. Она эквивалентна $\lambda_\wedge_\eta$ в смысле типизации: утверждение типизации, верное в системе  $\lambda_{\wedge \leqslant}$, верно в $\lambda_\wedge_\eta$ и наоборот. Поэтому в контексте нашей задачи эти две системы полностью взаимозаменяемы.

Определим отношение эквивалентности на типах следующим образом. 

\begin{definition}
$\sigma \sim \tau \iff \sigma \leqslant \tau$ и $\tau \leqslant \sigma$
\end{definition}

Благодаря правилу $(\leqslant)$, для эквивалентных типов верно следующее утверждение:

\begin{lemma}
Множества термов, тпизируемых эквивалентными типами, в точности совпадают.
\end{lemma}

В частности, это означает, что переход к эквивалентному типу не влияет на его обитаемость.

Легко видеть, что отношение <<$\sim$>> является отношением конгруэнтности (в смысле \cite{barendregt_2013}) а именно, верна следующая Лемма:
\begin{lemma}
Пусть $\pi, \rho, \sigma, \tau$ -- типы. 
Если $\pi \sim \rho$ и $\sigma \sim \tau$, то $(\pi \to \sigma) \sim (\rho \to \tau)$, а также $(\pi \wedge \sigma) \sim (\rho \wedge \tau)$.
\end{lemma}



Для удобства введём следующее обозначение: 
\begin{definition}
Для произвольного типа $\tau$ за $\hat{\tau}$ обозначим множество, полученное разделением всех пересечений на верхнем уровне $\tau$:
\begin{align*} 
\hat{\tau} = \{\tau_i \mid & \tau_1 \wedge \dots \wedge \tau_k = \tau \\
                               && \text{и $\tau_i$ не является пересечением ни для какого i} \}
\end{align*}
\end{definition}


В \cite{hindley_1982} показана следующая экивалентность: 
\begin{lemma} Для любых типов $\rho, \sigma, \tau$ $\colon$ 
$\rho \to \sigma \wedge \tau \sim (\rho \to \sigma) \wedge (\rho \to \tau)$
\end{lemma}

Применяя её многократно, мы можем привести тип к нормальной форме. А именно, введём операцию <<*>> следующим образом: 

\begin{definition}
\begin{align*}
\alpha^* &= \alpha \\
    &\text{если $\alpha$ --- типовая переменная }\\
(\sigma \wedge \tau)^* &= \sigma^* \wedge \tau^*\\
(\sigma \to \tau)^* &= \bigwedge \limits_{\varphi \in \hat{\tau^*}} \sigma \to \varphi
\end{align*}
\end{definition}

Например, $(\alpha \wedge \varepsilon \to \gamma \wedge (\beta \to \gamma \wedge \delta))^* = (\alpha \wedge \varepsilon \to \beta \to \gamma) \wedge (\alpha \wedge \varepsilon \to \beta \to \delta) \wedge (\alpha \wedge \varepsilon \to \gamma)$.

Операция <<*>> переводит тип в эквивалентный.

\begin{lemma} \label{normal form}
Для любого типа $\tau$: $\tau \sim \tau^*$
\end{lemma}


\begin{definition}
Тип $\tau$ находится в {\it нормальной форме}, если $\tau^* = \tau$.
\end{definition}

Общий вид типа в нормальной форме описывается следующей леммой:

\begin{lemma} \label{normal form form}
Для любого типа $\tau$: $\tau^* = \bigwedge \limits_i (\varphi_i_1 \to \cdots \to \varphi_i_{k i} \to \alpha_i$), где $\alpha_i$ -- типовая переменная.
\end{lemma}

\begin{corollary} \label{normal form unit}
Типы в нормальной форме без пересечений на верхнем уровне --- в точности типы вида $\dots \to \alpha$, где $\alpha$ -- некоторая типовая переменная.
\end{corollary}

\begin{definition} \label{typesize}
Мощностью типа назовём следующую величину: 
\begin{align*}
|\alpha| &= 1 \\
         &\text{если $\alpha$ --- типовая переменная }\\
|\rho_1 \to \rho_2 \to \cdots \to \rho_k \to \alpha| &= 2 + \Sigma\limits_{i = 1}^k |\rho_i|\\
                                                     &\text{если $\alpha$ --- типовая переменная и $k > 0$}\\
|\sigma \wedge \tau| &= 1 + |\sigma| + |\tau|\\
    &\text{если $\sigma \wedge \tau$ в нормальной форме}\\
|\tau| &= |\tau^*|\\
       &\text{если $\tau$ не в нормальной форме}
\end{align*}
\end{definition}

Используя индукцию и Лемму \ref{normal form form}, легко показать, что мощность типа корректно определена. Интуитивно, мощность типа $\tau$ ограничивает сверху количество типов-\hspace{0pt}подвыражений, которые участвуют в $\tau$ и которые могут быть рассмотрены описанным далее алгоритмом. Так, например, мощность $\alpha \to \beta \wedge \gamma$ равна $11$, типы-подвыражения при этом следующие: $(\alpha \to \beta) \wedge (\alpha \to \gamma), \alpha \to \beta, \alpha \to \gamma, \alpha (\times 2), \beta, \gamma$. По мощности типа удобно проводить индукцию, доказывающую свойства алгоритма.



В \cite{hindley_1982} показано, что отношение $(\leqslant)$ рекурсивно и существует алгоритм, позволяющий определить для двух типов $\alpha$ и $\beta$ верно ли, что $\alpha \leqslant \beta$:


\begin{algorithm} \label{subt algo}
\begin{align*}
    \alpha &\leqslant \beta  = (\alpha \ttt{ == } \beta) \\ 
                            &\text{если $\alpha$ и $\beta$ --- типовые переменные} \\
    \sigma &\leqslant \tau = False \\ 
                          &\text{если один из типов --- переменная, а другой тип стрелочный}\\
    (\sigma_1 \to \tau_1) &\leqslant (\sigma_2 \to \tau_2) = (\tau_1 \leqslant \tau_2) \ttt{ AND } (\sigma_2  \leqslant \sigma_1) \\
    \sigma &\leqslant \tau = \forall \tau_i \in \hat{\tau^*} \colon \exists \sigma_j \in \hat{\sigma^*} \colon \sigma_j \leqslant \tau_i
\end{align*}
\end{algorithm}

Введём ещё несколько обозначений. 

\begin{definition}
Обозначим множество всех подтипов $\tau$ через $\ub{\tau}$, а множество всех его надтипов через $\ob{\tau}$. То есть $\ub{\tau} = \{ \sigma \mid \sigma \leqslant \tau \}$, $\ob{\tau} =  \{ \sigma \mid \tau \leqslant \sigma \}$. 
\end{definition}

Далее будем считать, что надтипы и подтипы могут использоваться везде, где могут использоваться типы. При этом операции над типами <<поднимаются>> на уровень множеств. Так, например, выражение $\alpha \to \ob{\beta} \to \ub{\gamma}$ следует понимать как множество $\{ \alpha \to \beta' \to \gamma' \mid \beta' \in \ob{\beta}, \gamma' \in \ub{\gamma} \}$.

Утверждение о типизации, в котором фигурируют надтипы или подтипы, будем считать выводимым, если оно выводимо {\it для некоторых} элементов соответствующих можеств надтипов и подтипов.

Запись о вхождении типизации в контекст $(x \colon T) \in \Gamma$ в случае, если в $T$ фигурируют надтипы или подтипы, означает, что $(x \colon \tau) \in \Gamma$ {\it для некоторого} $\tau \in T$.

Запись вида $T_1 \leqslant T_2$, где в $T_i$ могут фигурировать подтипы и надтипы, означает, что $\forall \tau_1 \in T_1, \tau_2 \in T_2 : \tau_1 \leqslant \tau_2$.

\begin{lemma} \label{подтип стрелки}
Пусть $\sigma \to \tau$ -- тип в нормальной форме. Тогда
$\ub{\sigma \to \tau} = \varphi \wedge (\ob{\sigma} \to \ub{\tau})$,
для некоторого (возможно, пустого) $\varphi$.

\end{lemma}

\begin{proof}
    Включение $\varphi \wedge (\ob{\sigma} \to \ub{\tau}) \subseteq \ub{\sigma \to \tau}$ почти очевидно: 
    $$\infer[(A4)]{\varphi \wedge (\ob{\sigma} \to \ub{\tau}) \leqslant \sigma \to \tau}
                  {\infer[(R2)]{\ob{\sigma} \to \ub{\tau}}
                               {\sigma \leqslant \ob{\sigma} & \ub{\tau} \leqslant \tau}}$$
                               
    Для включения $\ub{\sigma \to \tau} \subseteq \varphi \wedge (\ob{\sigma} \to \ub{\tau})$ достаточно посмотреть на Алгоритм~\ref{subt algo}. Подтип $\sigma \to \tau$ --- это либо тип из $(\ob{\sigma} \to \ub{\tau})$ (третий случай алгоритма), либо тип-пересечение, один из элементов которого --- подтип $\sigma \to \tau$; остальные элементы можно <<переместить>> в $\varphi$ перестановкой.
\end{proof}


\subsection{Существенность $\eta$ правила}

Насколько добавление $\eta$-правила меняет систему? 
Система $\lambda_\wedge$ существенно слабее $\lambda_\wedge_\eta$, а именно, есть тип, необитаемый в системе без эта-правила и обитаемый в системе с ним. 

\begin{lemma} \label{существенность: типизация}
Утверждение о типизации $x : \alpha \to \beta \wedge \gamma \vdash x : \alpha \to \beta$ верно в $\lambda_{\wedge \eta}$ но неверно в $\lambda_{\wedge}$.
\end{lemma}

\begin{proof}
Докажем выводимость в системе $\lambda_{\wedge \leqslant}$:

$$\infer{x \colon \alpha \to \beta \wedge \gamma \vdash x \colon \alpha \to \beta}
        {\infer{x \colon \alpha \to \beta \wedge \gamma \vdash x \colon \alpha \to \beta \wedge \gamma} &
            \alpha \to \beta \wedge \gamma \leqslant \alpha \to \beta}
         $$ 
Чтобы доказать, что утверждение неверно в системе $\lambda_{\wedge}$, предположим обратное. Рассмотрим последнее правило вывода, которое могло быть применено, чтобы получить данное утверждение. $(I \to)$ и $(E \to)$ не могли быть применены, поскольку они типизируют абстракцию и аппликацию соответственно, а $x$ --- типовая переменная. Правило $(Ax)$ требует наличия соответствующей типизации в контексте, правило $(I \wedge)$ приписывает терму тип-пересечение, а не стрелочный тип.

Таким образом, единственным возможным правилом могло быть $(E \wedge)$:

$$\infer{x \colon \alpha \to \beta \wedge \gamma \vdash x \colon \alpha \to \beta}
        {\infer*{x \colon \alpha \to \beta \wedge \gamma \vdash x \colon (\alpha \to \beta) \wedge \sigma}}
$$ 
Какие правила вывода могли привести к полученному утверждению о типизации? Только $(I\wedge)$ и $(E\wedge)$. Однако легко видеть, что их <<обратное>> применение порождает в одной из веток утверждение вида $x \colon \alpha \to \beta \wedge \gamma \vdash x \colon (\alpha \to \beta) \wedge \sigma$ для некоторого (возможно, пустого) $\sigma$, но утверждение такого вида уже встречалось ранее, и, опять же, могло быть получено лишь с помощью $(I\wedge)$ и $(E\wedge)$. Значит, дерева вывода не существует.

\end{proof}

\begin{lemma}
Тип $\delta \wedge (\alpha \to \beta \wedge \gamma) \to \delta \wedge (\alpha \to \beta)$ пуст в $\lambda_{\wedge}$, но содержит терм \ttt{id} в $\lambda_{\wedge \eta}$. 
\end{lemma}

\begin{proof}

Анализом возможных деревьев вывода, аналогичным рассуждениям в предыдущей лемме, получаем, что для существования типизации $\vdash M \colon \delta \wedge (\alpha \to \beta \wedge \gamma) \to \delta \wedge (\alpha \to \beta)$ необходима и достаточна типизация $x \colon \alpha \to \beta \wedge \gamma \vdash x \colon \alpha \to \beta$. 

\end{proof}

\begin{notice}
    Тип  $(\alpha \to \beta \wedge \gamma) \to \alpha \to \beta$ обитаем в обеих системах. В $\lambda_{\wedge}$ он содержит $\lambda a b . a b$, но не $\lambda a . a$.
\end{notice}

\subsection{Выводы и результаты по главе}

В данной главе была рассмотрена система типов $\lambda_{\wedge \eta}$ и эквивалентная ей $\lambda_{\wedge \leqslant}$, для которых в следующих разделах будет приведён и формально обоснован населяющий алгоритм. Также было введено отношение подтипизации <<$\leqslant$>>; отношение эквивалентности <<$\sim$>>; операции $\ob{\tau}$ и $\ub{\tau}$, ставящие в соответствие типу множество его надтипов и подтипов соответственно;  операция <<$*$>>, приводящая тип к нормальной форме. Алгоритм приведения к нормальной форме является важной составной частью населяющего алгоритма, описанию которого посвящена следующая глава.


\end{document}
