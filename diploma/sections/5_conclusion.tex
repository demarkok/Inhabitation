\documentclass[../main.tex]{subfiles}
\begin{document}

В рамках данной работы были достигнуты следующие результаты: 

\begin{itemize}
    \item  Рассмотрены известные работы, относящиеся к родственным системам типов. 
    \item  Исправлены неточности статьи \cite{kusmierek_2007} в представленном в ней населяющем алгоритме для системы $\lambda_\wedge$ второго ранга. 
    \item  Разработан алгоритм для системы $\lambda_{\wedge \eta}$, основанный на алгоритме из \cite{kusmierek_2007}. Доказаны его корректность, полнота и завершаемость. Что доказывает, что задача обитаемости разрешима в системе типов $\lambda_\wedge_\eta$ второго ранга. 
    \item  Населяющий алгоритм для системы $\lambda_{\wedge \eta}$ реализован на языке Haskell \cite{kaysin_2019} 
\end{itemize}

Таким образом, основной результат данной работы -- доказательство разрешимости проблемы обитаемости в просто типизированном лямбда исчислении с пересечениями второго ранга и подтипизацией. Помимо разрешимости для задачи обитаемости в системе $\lambda_\wedge$ второго ранга была доказана её EXPTIME и EXPSPACE сложность. Есть основания полагать, что эти доказательства также адаптируются и к $\lambda_{\wedge \eta}$. 

\emptyline

Системы с пересечениями родственны чуть более известному семейству систем -- полиморфному лямбда исчислению (SystemF) и его модификациям, таким как SystemF$_\leqslant$. В этих системах вместо типового конструктора пересечения <<$\wedge$>> вводится конструктор <<$\forall$>>, который можно понимать как <<бесконечное пересечение>>. 

Близость этих систем подтверждается также в \cite{yokouchi_1995}, где доказывается <<вложение>> SystemF в $\lambda_\wedge$: SystemF эквивалентна некоторой подсистеме $\lambda_\wedge$ (которая обозначается $\lambda_\wedge^*$) в следующем смысле. Существует алгоритм $tr$, переводящий типы $\lambda_\wedge^*$ в типы SystemF, такой, что если терм $M$ типизируется типом $\tau$ в $\lambda_\wedge^*$, то $M$ типизируется типом $tr(\tau)$ в SystemF и наоборот -- если $M \colon \sigma$ в SystemF, то $\sigma = tr(\tau)$, где $M \colon \tau$ в $\lambda_\wedge^*$.

Таким образом, одним из основных направлений дальнейших исследований является изучение проблемы обитаемости в SystemF второго ранга. Обитаемость в теории типов эквивалентна выполнимости (доказуемости) в логике, а SystemF соответствует пропозициональной интуиционистской логике второго порядка. Поэтому эта проблема интересна не только специалистам в теории типов, но и математикам. Поскольку полиморфное лямбда исчисление устроено достаточно сложно даже с учётом ограничений на ранг, имеет смысл рассмотреть ещё более слабые подсистемы SystemF. Например, системы с ограниченным набором типовых переменных.


\end{document}