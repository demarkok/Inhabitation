\documentclass[../main.tex]{subfiles}
\begin{document} \label{sec:domain_review}

% \subsection{Обзор литературы}
Впервые типы были введены в логику Бертраном Расселом в начале XX века. Алонзо Чёрч ввёл их в лямбда исчисление в 1940 \cite{church_1940}, построив так называемое просто типизированное лямбда исчисление. Альтернативная версия этой системы была предложена Хаскелем Карри в 1969 \cite{curry_1969}. Системы Карри и Чёрча отличаются тем, что в системе Чёрча тип <<встроен>> в лямбда-терм. Это означает, что любой терм типизируется уникальным образом. В системе Карри один терм может иметь несколько типов, если они выводимы из аксиом и правил. Для задачи обитаемости различия этих двух систем не играют большой роли, поскольку тип всегда известен.

Системы типов с операцией пересечения получили своё развитие в начале 1980-х \cite{coppo_1980, coppo_1981}. Позднее Хиндли исследовал свойства этой системы и формально описал её семантику \cite{hindley_1982}. Кроме того, в этой статье было введено понятие {\it подтипизации} ($\leqslant$), означающее с теоретико-множественной точки зрения <<включение>> одного типа в другой, и доказаны некоторые его свойства. Подтипизация и его свойства активно используются в настоящей работе. 

Если говорить о проблеме обитаемости, первая работа, имеющая отношение к интересующей нас системе типов, появилась в 1979 году \cite{statman_1979}. В этой статье доказывается, что для просто типизированного лямбда исчисления (без пересечений) задача обитаемости разрешима и является PSPACE-полной. В 1981 была доказана неразрешимость задачи выполнимости в пропозициональной интуиционистской логике второго порядка \cite{gabbay_1981}. В соответствии с изоморфизмом Карри-Говарда, это означает неразрешимость задачи обитаемости в полиморфном лямбда исчислении (System F), где помимо типового конструктора <<$\to$>> вводится конструктор полиморфного типа <<$\forall$>>.

Полиморфизм просто типизированного лямбда исчисления c пересечениями более сильный: <<$\forall$>> можно рассматривать как бесконечное пересечение по всем возможным типам. В \cite{pottinger_1980} было показано, что в этой системе множество типизируемых термов и множество нормализуемых (имеющих нормальную форму) совпадают. 

Неразрешимость задачи обитаемости для системы с пересечениями была показана в \cite{urzyczyn_1999} (1999). % Ключевая идея доказательства -- сведение к обитаемости проблемы останова для конечного детерминированного автомата с очередью. 
Это означает, что данная система является слишком богатой в смысле выразительной мощности, и для существования алгоритмического решения задачи обитаемости необходимо наложить на систему некоторые ограничения. В \cite{leivant_1983} вводится понятие ранга типа -- меры его сложности. Конструкция, приведённая в \cite{urzyczyn_1999}, к населению которой сводится проблема останова, содержит типы ранга $\leqslant 4$. Поэтому ограничения <<ранг $\leqslant 4$>> недостаточно, чтобы сделать проблему разрешимой. 

В \cite{kurata_1995} было показана разрешимость задачи для другой версии системы с пересечениями и $\eta$ -- правилом (эквивалентно подтипизации). В этой системе нет ограничения на ранг типов; она ослабляется за счёт того, что в ней запрещено применять одно из правил вывода, а именно так называемое правило введения пересечения (\ref{i-wedge}). 

\begin{equation} \label{i-wedge}
\infer[(I\wedge)]{\Gamma \vdash M \colon \sigma \wedge \tau}{\Gamma \vdash M \colon \sigma & \Gamma \vdash M \colon \tau}
\end{equation}

В статье \cite{kusmierek_2007} Кушмерек доказал разрешимость задачи обитаемости в системе с пересечениями и ограничением <<ранг $\leqslant$ 2>>, предъявив экспоненциальный по памяти населяющий алгоритм и попутно доказав EXPTIME-трудность этой задачи. Настоящая работа во многом базируется на этой статье, в частности, алогритм, разработанный Кушмереком, лёг в основу Алгоритма~\ref{alg} для системы с подтипизацией, представленного далее. Алгоритм базируется на алгоритме Бена-Йелса, описанном в \cite{hindley_2008}. Но вместо решения одной задачи населенности, одновременно решается система из нескольких задач. Причиной этой множественности являются пересечения: найти терм, имеющий тип $\sigma \wedge \tau$ означает найти терм, который одновременно может быть типизирован и как $\sigma$, и как $\tau$. 

Вопрос о том, достаточно ли ограничения <<ранг $\leqslant 3$>> для разрешимости задачи населённости оставался открытым до 2009 года. Статья Уржицына\cite{urzyczyn_2009} закрывает промежуток между <<ранг $\leqslant 2$ и <<ранг $\leqslant 4$ и доказывает, что в системе пересечениями ранга $3$ задача всё ещё неразрешима. Кроме того, в статье доказывается EXPSPACE-трудность задачи для типов ранга 2, что в купе с алгоритмам Кушмерека обеспечивает её EXPSPACE-полноту.







% \subsection{Просто типизированное лямбда исчисление}

% \subsubsection{Лямбда термы}
% \subsubsection{Типизация}

\end{document}