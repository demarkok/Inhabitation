\documentclass[../main.tex]{subfiles}
\begin{document}

Почти во всех языках программирования в том или ином виде представлены типы. Чем сильнее система типов языка, тем больше ограничений она накладывает на корректные программы и тем больше ошибок может быть найдено на {\it этапе компиляции}, что позволяет уменьшить число потенциальных ошибок на {\it этапе исполнения}, когда программа работает в реальной среде, и цена каждой ошибки велика. Изучением систем типов и их свойств занимается теория типов. 

Помимо языков программирования, теория типов применяется в системах интерактивных доказательств -- средствах, позволяющих математикам полуавтоматически формулировать и доказывать теоремы на некотором формальном языке; используется для описания оснований математики как альтернатива наивной теории множеств.

Наиболее развитую систему типов имеют функциональные языки программирования. Основанием этой системы является типизированное {\it лямбда исчисление}.

Лямбда исчисление было впервые введено Алонзо Чёрчем в 1930-х годах и на данный момент имеет множество модификаций. Самая базовое из типизированных лямбда исчислений -- просто типизированное лямбда исчисление. Типы в этой системе конструируются из двух составных частей: типовых переменных (например, \ttt{Int} -- тип целых чисел или \ttt{String} -- тип строк) и операции (типового конструктора) <<$\to$>>. <<$\to$>> представляет функциональный тип. Например, тип <<\ttt{Int} $\to$ \ttt{String}>> -- это тип функций, отображающих целые числа в строки. 

Система типов, рассматриваемая в данной работе, содержит ещё один типовой конструктор -- <<$\wedge$>>, представляющий пересечение двух типов. Лямбда исчисление с пересечениями известно с $1960$-х, значимость этой системы подтверждается многочисленными работами, посвященными ей: модели лямбда исчисления \cite{alessi_2006, coppo_1980}, проблемы нормализации и оптимальной редукции \cite{neergaard_2004, pottinger_1980}, вывод типов и компиляция \cite{kfoury_2004, wells_2002}.

Другим важным аспектом систем типов с пересечениями является их логическая интерпретация. Изоморфизм Карри-Говарда ставит в соответствие типам логические высказывания, а термам -- доказательства этих высказываний. 
Благодаря такой интерпретации, можно считать, что решая задачу обитаемости, мы по высказыванию изучаем его доказательства -- ищем их или проверяем их наличие.
Просто типизированное лямбда исчисление с пересечениями при этом соответствует пропозициональной логике с пересечениями. Неразрешимость задачи обитаемости в этой системе \cite{urzyczyn_1999} говорит о её большой выразительной силе. Поэтому, в частности, важно разграничить случаи разрешимости и неразрешимости. В данной работе мы рассматриваем просто типизированное лямбда исчисление с пересечениями и подтипизацией. При этом выясняется, что, ограничивая определённым образом множество допустимых типов системы, мы ослабляем её, и задача обитаемости становится разрешимой.



\end{document}