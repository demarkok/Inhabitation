% В этом шаблоне используется класс spbau-diploma. Его можно найти и, если требуется, 
% поправить в файле spbau-diploma.cls

\documentclass{spbau-diploma}
\begin{document}
% Год, город, название университета и факультета предопределены,
% но можно и поменять.
% Если англоязычная титульная страница не нужна, то ее можно просто удалить.
\filltitle{ru}{
    % chair              = {Кафедра математических и информационных технологий},
    title              = {Задача обитаемости в системах типов низкого ранга},
    % Здесь указывается тип работы. Возможные значения:
    %   coursework - Курсовая работа
    %   diploma - Диплом специалиста
    %   master - Диплом магистра
    %   bachelor - Диплом бакалавра
    type               = {bachelor},
    position           = {студента},
    group              = 666,
    author             = {Кайсин Илья Сергеевич},
    supervisorPosition = {к.\,ф.-м.\,н., профессор},
    supervisor         = {Москвин Д.\,Н.},
    reviewerPosition   = {},
    reviewer           = {Пеленицын А.\,М},
    chairHeadPosition  = {д.\,ф.-м.\,н., профессор},
    chairHead          = {Омельченко А.\,В.},
    % university = {САНКТ-ПЕТЕРБУРГСКИЙ АКАДЕМИЧЕСКИЙ УНИВЕРСИТЕТ},
    % faculty = {Центр высшего образования},
    % city = {Санкт-Петербург},
    % year             = {2013}
}


\maketitle
\tableofcontents
% У введения нет номера главы

\section*{Аннотация}
\subfile{sections/0_abstract}

\section*{Введение}
\subfile{sections/1_introduction}

\section{Обзор предметной области}
\subfile{sections/1_a_domain_review}

\section{Система типов $\lambda_\wedge_\eta$}
\subfile{sections/2_the_system}

\section{Населяющий алгоритм}
\subfile{sections/3_the_algorithm}

\section{Свойства алгоритма}
\subfile{sections/4_properties}

\bibliographystyle{ugost2008ls}
\bibliography{diploma.bib}
\end{document}
