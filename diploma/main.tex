% В этом шаблоне используется класс spbau-diploma. Его можно найти и, если требуется, 
% поправить в файле spbau-diploma.cls

\documentclass{spbau-diploma}

\usepackage{amsmath}
\usepackage{subfiles}
\usepackage{graphicx}
\usepackage[colorinlistoftodos]{todonotes}
\usepackage[colorlinks=true, allcolors=blue]{hyperref}
\usepackage{multicol}
\usepackage{stmaryrd}
\usepackage{proof}
\usepackage{amsthm}
\usepackage{datetime}
\usepackage{amssymb}
\usepackage{cite}
\usepackage{pdfpages}

\newdateformat{monthyeardate}{%
  \monthname[\THEMONTH], \THEYEAR}

\newtheorem{theorem}{Теорема}[section]
\newtheorem{lemma}[]{Лемма}[section]
\newtheorem{corollary}{Следствие}[lemma]
\newtheorem{proposition}[]{Предложение}
\newtheorem{algorithm}{Алгоритм}[section]
\newtheorem{notice}[]{Замечание}[lemma]
\newtheorem{definition}{Определение}[section]

\newcommand{\ttt}[1] {\texttt{#1}}
\newcommand{\ob}[1]{\mkern 1.5mu\overline{\mkern-1.5mu#1\mkern-1.5mu}\mkern 1.5mu}
\newcommand{\ub}[1] {\mkern 1.5mu\underline{\mkern-1.5mu#1\mkern-1.5mu}\mkern 1.5mu}
\newcommand{\emptyline} {\vspace{\baselineskip}}


\begin{document}
% Год, город, название университета и факультета предопределены,
% но можно и поменять.
% Если англоязычная титульная страница не нужна, то ее можно просто удалить.
\filltitle{ru}{
    % chair              = {Кафедра математических и информационных технологий},
    title              = {ЗАДАЧА ОБИТАЕМОСТИ В СИСТЕМАХ ТИПОВ НИЗКОГО РАНГА},
    % Здесь указывается тип работы. Возможные значения:
    %   coursework - Курсовая работа
    %   diploma - Диплом специалиста
    %   master - Диплом магистра
    %   bachelor - Диплом бакалавра
    type               = {bachelor},
    position           = {студента},
    group              = 666,
    author             = {Кайсин Илья Сергеевич},
    supervisorPosition = {к.\,ф.-м.\,н., доцент},
    supervisor         = {Москвин Денис Николаевич},
    reviewerPosition   = {},
    reviewer           = {Пеленицын А.\,М},
    chairHeadPosition  = {д.\,ф.-м.\,н., профессор},
    chairHead          = {Омельченко А.\,В.},
    % university = {САНКТ-ПЕТЕРБУРГСКИЙ АКАДЕМИЧЕСКИЙ УНИВЕРСИТЕТ},
    % faculty = {Центр высшего образования},
    % city = {Санкт-Петербург},
    % year             = {2013}
}


\maketitle
\tableofcontents
% У введения нет номера главы

\section*{Аннотация}
\subfile{sections/0_abstract}

\section*{Введение}
\subfile{sections/1_introduction}

\section{Обзор предметной области}
\subfile{sections/1_a_domain_review}

\section{Система типов $\lambda_{\wedge \eta}$}
\subfile{sections/2_the_system}

\section{Населяющий алгоритм}
\subfile{sections/3_the_algorithm}

\section{Свойства алгоритма}
\subfile{sections/4_properties}

\section*{Заключение}
\subfile{sections/5_conclusion}


\bibliographystyle{ugost2008my}
\bibliography{diploma.bib}
\end{document}
